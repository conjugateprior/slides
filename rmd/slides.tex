\documentclass[11pt,professionalfonts,t]{beamer}
\usetheme{metropolis}

\usepackage[]{graphicx}
\usepackage[]{color}
\usepackage{mathspec}
\setsansfont[BoldItalicFont={* Demi Bold Italic},
             ItalicFont={* Italic},
             BoldFont={* Demi Bold},
             UprightFont={* Regular}]{Avenir Next}
\setmonofont[Scale = MatchLowercase,Mapping=tex-ansi]{Inconsolata}
\setmathfont(Digits,Latin,Greek){Avenir Next}
\definecolor{pton}{HTML}{EE7F2D}

\setbeamercolor{frametitle}{fg=white, bg=pton}
\metroset{progressbar=foot,numbering=none}
\setbeamercolor{progress bar in head/foot}{fg=pton,bg=darkgray}


% used by stargazer, texreg, etc.
\usepackage{ctable}
\usepackage{hyperref}
\usepackage{dcolumn}
\usepackage{booktabs}

% for graphical models
\usepackage{tikz}
\usetikzlibrary{arrows}
\usetikzlibrary{bayesnet} % requires download
\tikzstyle{const} = [rectangle, inner sep=4pt, node distance=1]
\newcommand{\spurious}[3][]{ % \edge [options] {inputs} {outputs}
  % Connect all nodes #2 to all nodes #3.
  \foreach \x in {#2} { %
    \foreach \y in {#3} { %
      \path (\x) edge [-,#1,dotted] (\y) ;%
      %\draw[->,#1] (\x) -- (\y) ;%
    };
  };
}


% I dislike bullet points
\usepackage[shortlabels]{enumitem}
\setlist[itemize]{label={}}
% switch off the list tightener bc it looks better when we have no bullets
\providecommand{\tightlist}{}%\setlength{\itemsep}{0pt}}%\setlength{\parskip}{0pt}}

\IfFileExists{upquote.sty}{\usepackage{upquote}}{}

\usepackage{biblatex}


\usepackage[labelformat=empty]{caption}



% Comment these out if you don't want a slide with just the
% part/section/subsection/subsubsection title:
\AtBeginPart{
  \let\insertpartnumber\relax
  \let\partname\relax
  \frame{\partpage}
}
\AtBeginSection{
  \let\insertsectionnumber\relax
  \let\sectionname\relax
  \frame{\sectionpage}
}
\AtBeginSubsection{
  \let\insertsubsectionnumber\relax
  \let\subsectionname\relax
  \frame{\subsectionpage}
}

\setlength{\parindent}{0pt}
\setlength{\parskip}{6pt plus 2pt minus 1pt}
\setlength{\emergencystretch}{3em}  % prevent overfull lines
\setcounter{secnumdepth}{0}


\title{WWS 403 Methods Lab}
\subtitle{Lecture 3}
\author{Will Lowe}
\institute{Department of Politics, Princeton}
\date{}

\begin{document}
\frame{\titlepage}

\begin{frame}{Tonight}

The Plan

\begin{itemize}
\tightlist
\item
  Announcements: Brief debrief and extra sessions
\item
  Part 1: How to get people to vote?
\item
  Part 2: What should I put in my model? \(\beta\)
\item
  Lab: Because we love labs
\end{itemize}

\end{frame}

\begin{frame}{Extras: Selection on the dependent variable}

Even when there is no causal link between X and Y

\begin{center}
{\large
\tikz{ % pure collision
  \node[const]                               (z) {$\text{Z}$};
  \node[const, above=of z, xshift=-1.2cm] (x) {$\text{X}$};
  \node[const, above=of z, xshift=1.2cm]  (y) {$\text{Y}$};
  \edge {x,y} {z} ; %
  }
}
\end{center}

conditioning on Z will create spurious association.

\begin{center}
{\large
\tikz{ % pure collision
  \node[const]                               (z) {$\text{{Z}}$};
  \node[const, above=of z, xshift=-1.2cm] (x) {$\text{X}$};
  \node[const, above=of z, xshift=1.2cm]  (y) {$\text{Y}$};
  \edge {x,y} {z} ; %
  \spurious[thick]{x}{y}
  }
}
\end{center}

\end{frame}

\begin{frame}{Selection on the dependent variable}

This is why selecting cases on the dependent variable is problematic.

\pause

Assume Z raises Y but does not affect X

\begin{center}
{\large
\tikz{ % measurement error
  \node[const]                            (z) {$\text{{Z}}$};
  \node[const, below=of z, xshift=-1.2cm] (x) {$\text{X}$};
  \node[const, below=of z, xshift=1.2cm]  (y) {$\text{Y}$};
  \edge [thick,draw=pton] {x} {y} ; %
  \edge {z} {y} ;
  }
}
\end{center}

Z is not a confounder

\end{frame}

\begin{frame}{Selection on the dependent variable}

However, if we select (S) only units with, say, high Y

\begin{center}
\tikz{ % selection on the dependent variable
  \node[const]                               (z) {$\text{{Z}}$};
  \node[const, below=of z, xshift=-1.2cm] (x) {$\text{X}$};
  \node[const, below=of z, xshift=1.2cm]  (y) {$\text{Y}$};
  \node[const, right=1.2cm of y]            (s) {$\text{S}$};
  \edge [thick,draw=pton]{x} {y} ; %
  \edge {z}{y};
  \edge {y} {s} ; %
}
\end{center}

Then we will see spurious association between X and Z

\begin{center}
\tikz{ % selection on the dependent variable
  \node[const]                               (z) {$\text{{Z}}$};
  \node[const, below=of z, xshift=-1.2cm] (x) {$\text{X}$};
  \node[const, below=of z, xshift=1.2cm]  (y) {$\text{Y}$};
  \node[const, right=1.2cm of y]            (s) {$\text{S}$};
  \edge [thick,draw=pton]{x} {y} ; %
  \edge {z}{y}
  \edge {y} {s} ; %
  \spurious[thick]{x}{z}
}
\end{center}

\end{frame}

\begin{frame}{Selection on the dependent variable}

\begin{center}
\tikz{ % selection on the dependent variable
  \node[const]                               (z) {$\text{{Z}}$};
  \node[const, below=of z, xshift=-1.2cm] (x) {$\text{X}$};
  \node[const, below=of z, xshift=1.2cm]  (y) {$\text{Y}$};
  \node[const, right=1.2cm of y]            (s) {$\text{S}$};
  \edge [thick,draw=pton]{x} {y} ; %
  \edge {z}{y}
  \edge {y} {s} ; %
  \spurious[thick]{x}{z}
}
\end{center}

This subsample will have more high Z units with low X and more low Z
units with high X than the population.

\pause

So effect of X on Y will seem to be \emph{larger} than it really is.

\end{frame}

\begin{frame}{Selection on the dependent variable}

This effect occurs regardless of whether Z is measured.

\pause

Assume Z is \emph{not} measured. Then it is part of the error term in
the regression \[
Y ~=~ \alpha + X \beta + \epsilon 
\]

\pause

\begin{center}
\tikz{ % selection on the dependent variable
  \node[const]                               (z) {$\epsilon$};
  \node[const, below=of z, xshift=-1.2cm] (x) {$\text{X}$};
  \node[const, below=of z, xshift=1.2cm]  (y) {$\text{Y}$};
  \node[const, right=1.2cm of y]            (s) {$\text{S}$};
  \edge [thick,draw=pton]{x} {y} ; %
  \edge {z}{y}
  \edge {y} {s} ; %
  \spurious[thick]{x}{z}
}
\end{center}

\(\epsilon\) is now correlated with X because of the selection, so
\(\hat{\beta}\) no longer estimates \(\beta\).

\end{frame}


\end{document}
