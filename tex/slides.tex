\documentclass{beamer}
\usetheme{metropolis}
\usefonttheme{professionalfonts} % required for mathspec \usepackage{mathspec}

\usepackage{mathspec}
\defaultfontfeatures{Mapping=tex-text}
\setsansfont[BoldItalicFont={* Demi Bold Italic}, 
             ItalicFont={* Italic}, 
             BoldFont={* Demi Bold}, 
             UprightFont={* Regular}]{Avenir Next}
\setmonofont[Scale = MatchLowercase,Mapping=tex-ansi]{Inconsolata}
\setmathfont(Digits,Latin,Greek){Avenir Next}

\definecolor{pton}{HTML}{EE7F2D}
\definecolor{pblue}{HTML}{2d9cee}
\setbeamercolor{frametitle}{fg=white, bg=pton}
\metroset{progressbar=foot,numbering=none,titleformat frame=allsmallcaps}

\title{}
\date{\today} \author{Will Lowe}
\institute{Department of Politics, Princeton University} 

\usepackage{ctable}
\usepackage{hyperref}
\usepackage{dcolumn}
\usepackage{booktabs}

\newcommand{\ita}{\begin{itemize}}
\newcommand{\itm}{\item[]}
\newcommand{\itz}{\end{itemize}}

\usepackage{tikz}
\usetikzlibrary{arrows}
\usetikzlibrary{bayesnet}
\tikzstyle{const} = [rectangle, inner sep=4pt, node distance=1]
\newcommand{\spurious}[3][]{ % \edge [options] {inputs} {outputs}
  % Connect all nodes #2 to all nodes #3.
  \foreach \x in {#2} { %
    \foreach \y in {#3} { %
      \path (\x) edge [-,#1,dotted] (\y) ;%
      %\draw[->,#1] (\x) -- (\y) ;%
    };
  };
}

\title{WWS 403 Methods Lab}
\subtitle{Lecture 3}
\author{Will Lowe}
\institute{Department of Politics, Princeton University}
\date{}

\begin{document}

\maketitle

\begin{frame}[t,fragile]\frametitle{Tonight}

The Plan
\ita
\itm Announcements: Brief debrief and extra sessions
\itm Part 1: How to get people to vote?
\itm Part 2: What should I put in my model? $\beta$
\itm Lab: Because we love labs
\itz

\end{frame}

\begin{frame}[t,fragile]\frametitle{Extras: Selection on the dependent variable}

Even when there is no causal link between X and Y
\begin{center}
{\large
\tikz{ % pure collision
  \node[const]                               (z) {$\text{Z}$};
  \node[const, above=of z, xshift=-1.2cm] (x) {$\text{X}$};
  \node[const, above=of z, xshift=1.2cm]  (y) {$\text{Y}$};
  \edge {x,y} {z} ; %

}
}\end{center}
conditioning on Z will create spurious association.
\begin{center}
{\large
\tikz{ % pure collision
  \node[const]                               (z) {$\text{{Z}}$};
  \node[const, above=of z, xshift=-1.2cm] (x) {$\text{X}$};
  \node[const, above=of z, xshift=1.2cm]  (y) {$\text{Y}$};
  \edge {x,y} {z} ; %
  \spurious[thick,draw=pblue]{x}{y}

}
}\end{center}


\end{frame}

\begin{frame}[t,fragile]\frametitle{Selection on the dependent variable}

This is why selecting cases on the dependent variable is problematic.

\pause

Assume Z raises Y but does not affect X
\begin{center}
{\large
\tikz{ % measurement error
  \node[const]                            (z) {$\text{{Z}}$};
  \node[const, below=of z, xshift=-1.2cm] (x) {$\text{X}$};
  \node[const, below=of z, xshift=1.2cm]  (y) {$\text{Y}$};
  \edge [thick,draw=pton] {x} {y} ; %
  \edge {z} {y} ;
}
}\end{center}
Z is not a confounder

\end{frame}

\begin{frame}[t,fragile]\frametitle{Selection on the dependent variable}

However, if we select (S) only units with, say, high Y
\begin{center}
\tikz{ % selection on the dependent variable
  \node[const]                               (z) {$\text{{Z}}$};
  \node[const, below=of z, xshift=-1.2cm] (x) {$\text{X}$};
  \node[const, below=of z, xshift=1.2cm]  (y) {$\text{Y}$};
  \node[const, right=1.2cm of y]            (s) {$\text{S}$};
  \edge [thick,draw=pton]{x} {y} ; %
  \edge {z}{y};
  \edge {y} {s} ; %
}
\end{center}
Then we will see spurious association between X and Z
\begin{center}
\tikz{ % selection on the dependent variable
  \node[const]                               (z) {$\text{{Z}}$};
  \node[const, below=of z, xshift=-1.2cm] (x) {$\text{X}$};
  \node[const, below=of z, xshift=1.2cm]  (y) {$\text{Y}$};
  \node[const, right=1.2cm of y]            (s) {$\text{S}$};
  \edge [thick,draw=pton]{x} {y} ; %
  \edge {z}{y}
  \edge {y} {s} ; %
  \spurious[thick,draw=pblue]{x}{z}
}
\end{center}

\end{frame}

\begin{frame}[t,fragile]\frametitle{Selection on the dependent variable}

\begin{center}
\tikz{ % selection on the dependent variable
  \node[const]                               (z) {$\text{{Z}}$};
  \node[const, below=of z, xshift=-1.2cm] (x) {$\text{X}$};
  \node[const, below=of z, xshift=1.2cm]  (y) {$\text{Y}$};
  \node[const, right=1.2cm of y]            (s) {$\text{S}$};
  \edge [thick,draw=pton]{x} {y} ; %
  \edge {z}{y}
  \edge {y} {s} ; %
  \spurious[thick,draw=pblue]{x}{z}
}
\end{center}
This subsample will have more high Z units with low X and more low Z units with high X than the population.

\pause

So effect of X on Y will seem to be \textit{larger} than it really is.

\end{frame}

\begin{frame}[t,fragile]\frametitle{Selection on the dependent variable}

This effect occurs regardless of whether Z is measured.

\pause

Assume Z is \textit{not} measured.  Then it is part of the error term in the  regression
\begin{equation*}
Y ~=~ \alpha + X \beta + \epsilon 
\end{equation*}

\pause 

\begin{center}
\tikz{ % selection on the dependent variable
  \node[const]                               (z) {$\epsilon$};
  \node[const, below=of z, xshift=-1.2cm] (x) {$\text{X}$};
  \node[const, below=of z, xshift=1.2cm]  (y) {$\text{Y}$};
  \node[const, right=1.2cm of y]            (s) {$\text{S}$};
  \edge [thick,draw=pton]{x} {y} ; %
  \edge {z}{y}
  \edge {y} {s} ; %
  \spurious[thick,draw=pblue]{x}{z}
}
\end{center}

$\epsilon$ is now correlated with X because of the selection, so $\hat{\beta}$ no longer estimates $\beta$.


\end{frame}
\end{document}
